\documentclass[12pt,a4paper]{article}
\usepackage[utf8]{inputenc}
\usepackage{amsmath,amssymb,amsthm}
\usepackage{geometry}
\usepackage{booktabs}
\usepackage{longtable}
\usepackage{hyperref}
\usepackage{enumitem}

\geometry{
    left=2.5cm,
    right=2.5cm,
    top=2.5cm,
    bottom=2.5cm,
}

\hypersetup{
    colorlinks=true,
    linkcolor=blue,
    filecolor=magenta,      
    urlcolor=cyan,
}

\title{Manual Intervention and Expert Input Guide}
\author{Shaikh \& Tonak (1994) Replication and Extension}
\date{\today}

\begin{document}

\maketitle

\begin{abstract}
This document outlines the key areas in the replication and extension of Shaikh \& Tonak (1994) that require manual intervention, expert judgment, or methodological choices. While the project strives for maximum automation and fidelity, certain aspects of the data and methodology necessitate human input.
\end{abstract}

\section{Historical Replication (Phase 1)}

\subsection{Handling of the 1973 Capacity Utilization Gap}
\begin{itemize}
    \item \textbf{Issue:} The original book data for capacity utilization (`u`) shows a value of 0.0 for the year 1973. This is a data anomaly that makes the profit rate calculation mathematically impossible for that year.
    \item \textbf{Intervention:} An expert decision was made to handle this anomaly.
    \item \textbf{Action Taken:} The value for 1973 was interpolated as the midpoint between the 1972 and 1974 values. This is a methodological choice to create a continuous series, but it's important to note that this is an alteration of the raw data. The original `u=0.0` is preserved in the source data files for full transparency.
\end{itemize}

\section{Modern Extension (Phase 2)}

\subsection{Industry Classification Correspondence (SIC to NAICS)}
\begin{itemize}
    \item \textbf{Issue:} The historical data is based on the Standard Industrial Classification (SIC) system, while modern data uses the North American Industry Classification System (NAICS).
    \item \textbf{Intervention:} A mapping between SIC and NAICS codes is required to align the historical and modern datasets.
    \item \textbf{Action Taken:} A correspondence file (e.g., `config/industry_correspondences/st_naics_correspondence.json`) was created. This file requires expert review, especially for broad categories like "Services," where the mapping can be ambiguous.
\end{itemize}

\subsection{Assumptions for Modern Surplus Product (SP)}
\begin{itemize}
    \item \textbf{Issue:} A direct, one-to-one modern equivalent of the book's SP calculation is not available due to changes in NIPA accounting.
    \item \textbf{Intervention:} An assumption is needed to scale a modern data series (like corporate profits) to be comparable to the historical SP.
    \item \textbf{Action Taken:} A default annual nominal growth assumption (e.g., 3\%) is used to scale modern corporate profits. This is a significant modeling choice and should be treated as an expert-editable parameter.
\end{itemize}

\subsection{Estimation of Modern Capital Stock (K)}
\begin{itemize}
    \item \textbf{Issue:} Similar to SP, a direct modern equivalent of the book's capital stock series (`K` and `KK`) is not straightforward.
    \item \textbf{Intervention:} A method must be chosen to estimate the modern capital stock in a way that is consistent with the historical data.
    \item \textbf{Action Taken:} The modern capital stock is estimated based on the median historical ratio of Capital to Surplus Product (K/SP). This is a reasonable but coarse approximation and represents another key expert-editable parameter.
\end{itemize}

\subsection{Definitional Ambiguities}
\begin{itemize}
    \item \textbf{Issue:} Some variables in the book, like the capital growth rate (`gK`), are not defined with sufficient precision to be unambiguously replicated.
    \item \textbf{Intervention:} An interpretation of the likely definition is required, based on the available text and data.
    \item \textbf{Action Taken:} The replication proceeds with the most plausible interpretation (e.g., that `gK` is based on `I/K`), but this is flagged as an area where the replication may not be "perfect" until more precise definitions are discovered.
\end{itemize}

\section{Conclusion}
The final dataset is a product of both automated, faithful replication and necessary, documented human interventions. This guide serves as a transparent record of these interventions, allowing future researchers to understand the methodological choices made and to explore alternative assumptions by modifying the relevant configuration files and parameters.

\end{document}
