\documentclass[12pt,a4paper]{article}
\usepackage[utf8]{inputenc}
\usepackage{amsmath,amssymb,amsthm}
\usepackage{geometry}
\usepackage{booktabs}
\usepackage{longtable}
\usepackage{enumitem}
\usepackage{hyperref}
\usepackage{cleveref}
\usepackage{tcolorbox}
\usepackage{listings}
\usepackage{xcolor}
\usepackage{tabularx}

\definecolor{codegreen}{rgb}{0,0.6,0}
\definecolor{codegray}{rgb}{0.5,0.5,0.5}
\definecolor{codepurple}{rgb}{0.58,0,0.82}
\definecolor{backcolour}{rgb}{0.95,0.95,0.92}

\lstdefinestyle{mystyle}{
    backgroundcolor=\color{backcolour},   
    commentstyle=\color{codegreen},
    keywordstyle=\color{magenta},
    numberstyle=\tiny\color{codegray},
    stringstyle=\color{codepurple},
    basicstyle=\footnotesize\ttfamily,
    breakatwhitespace=false,         
    breaklines=true,                 
    captionpos=b,                    
    keepspaces=true,                 
    numbers=left,                    
    numbersep=5pt,                  
    showspaces=false,                
    showstringspaces=false,
    showtabs=false,                  
    tabsize=2,
    breaklines=true,
    postbreak=\mbox{\textcolor{red}{$\hookrightarrow$}\space},
}

\lstset{style=mystyle}

\geometry{
    left=2.5cm,
    right=2.5cm,
    top=2.5cm,
    bottom=2.5cm,
    heightrounded, 
}

\hypersetup{
    colorlinks=true,
    linkcolor=blue,
    filecolor=magenta,      
    urlcolor=cyan,
}

\title{Shaikh \& Tonak (1994) Code Explainer}
\author{Methodology, Formula, and Code Side-by-Side}
\date{\today}

\begin{document}

\maketitle

\begin{abstract}
This document provides a detailed breakdown of the key formulas and methodologies in Shaikh \& Tonak (1994), showing the original book descriptions, the mathematical formulation in LaTeX, and the corresponding Python implementation.
\end{abstract}

\section{Unified Capital Stock (K)}

\paragraph{Book Description} The book uses specific capital stock measures (KK for 1958--1973, K for 1974--1989) without interpolation between periods.

\begin{equation*}
K_t = \begin{cases}
KK_t & \text{if } t \leq 1973 \\
K_t & \text{if } t \geq 1974
\end{cases}
\end{equation*}

\begin{tcolorbox}[colback=gray!5!white,colframe=gray!60!black,title=Python Implementation]
\begin{lstlisting}[language=Python]
def create_unified_capital_series(self, df):
    K_unified = pd.Series(index=df.index, dtype=float, name='K_unified')
    # Use KK for 1958-1973
    for year in df.index:
        if year <= 1973:
            K_unified.loc[year] = df.loc[year, 'KK']
    # Use K for 1974-1989
    for year in df.index:
        if year >= 1974:
            K_unified.loc[year] = df.loc[year, 'K']
    return K_unified
\end{lstlisting}
\end{tcolorbox}

\section{Profit Rate (r)}

\paragraph{Book Description} The profit rate r' is reproduced with very small error using \( r_t = SP_t/(K_t\,u_t) \). This suggests the operational definition in Table 5.4 matches an SP-based construction, not the textbook \( s'/(1+c') \) identity.

\begin{equation*}
r_t = \frac{SP_t}{K_t \times u_t}
\end{equation*}

\begin{tcolorbox}[colback=gray!5!white,colframe=gray!60!black,title=Python Implementation]
\begin{lstlisting}[language=Python]
def calculate_marxian_profit_rate(self, df):
    SP = df.get('SP')
    K_unified = self.create_unified_capital_series(df)
    u = df.get('u')

    denominator = K_unified * u
    mask = (SP.notna()) & (denominator != 0)
    r_exact = pd.Series(index=df.index, dtype=float)
    r_exact.loc[mask] = SP.loc[mask] / denominator.loc[mask]

    return r_exact
\end{lstlisting}
\end{tcolorbox}

\section{Organic Composition of Capital (c')}

\paragraph{Book Description} The book's tables provide the organic composition of capital directly as c'. The replication uses these values without recalculation.

\begin{equation*}
q_t = c'_t
\end{equation*}

\begin{tcolorbox}[colback=gray!5!white,colframe=gray!60!black,title=Python Implementation]
\begin{lstlisting}[language=Python]
def calculate_organic_composition(self, df):
    c_exact = df.get("c'")
    return c_exact
\end{lstlisting}
\end{tcolorbox}

\section{Rate of Surplus Value (s')}

\paragraph{Book Description} Similar to the organic composition, the rate of surplus value is provided directly in the book's tables as s'.

\begin{equation*}
s'_t = \frac{S_t}{V_t} \quad \text{(from book data)}
\end{equation*}

\begin{tcolorbox}[colback=gray!5!white,colframe=gray!60!black,title=Python Implementation]
\begin{lstlisting}[language=Python]
def calculate_surplus_value_rate(self, df):
    svv_exact = df.get("s'")
    return svv_exact
\end{lstlisting}
\end{tcolorbox}

\end{document}
